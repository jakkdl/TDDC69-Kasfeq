\begin{center}
   \textbf{\Huge Planering}\\[1cm]
\end{center}
\section{Introduktion till projektet}
Kasfeq är ett multiplayer, modulärt, realtids fighting-spel som drar inspirationer från spel såsom Super Smash Bros. och LieroXtreme.\\
Grundprincipen med spelet är att besegra alla motståndare. Detta görs med hjälp av vapen, spells, powerups och melee. Man spelar på en bana som kan vara dynamisk, t.ex. förstörbar terräng eller dynamiska element, eller statisk.\\
\vspace{11pt}
Spelet har inte nätverksstöd utan spelas simultant vid en dator med hjälp av en eller flera tangentbord och kontroller.\\
\vspace{11pt}
Spelet är modulärt och detta möjliggör stor variation och hög underhållningsfaktor. Man kan t.ex. ändra/inaktivera många av spelets funktioner, såsom gravitation, friktion, explosioner, grafik, banan, etc.\\

\section{Ytterligare bakgrundsinformation}
Vi kommer använda oss utav OpenGL (Genom LWJGL biblioteket http://www.lwjgl.org/). Vi vet dock inte om OpenGL kommer att fungera på IDA's Solarisklienter, isf så kommer vi redovisa på medtagna bärbara datorer eller på annat sätt.\\
\vspace{11pt}
Eftersom spelet ska vara modulärt så kommer vi använda oss av klassen java.util.ServiceLoader. Denna klass tillåter oss att ladda in externa klasser i efter hand.\\
\vspace{11pt}
http://docs.oracle.com/javase/6/docs/api/java/util/ServiceLoader.html
\section{Milstolpar}
\section{Övriga implementationsförberedelser}
\section{Utveckling och samarbete}
Vi kommer använda GIT för versionshantering då vi har stor erfarenhet av att använda det. Vi kommer arbeta mycket på kvällar/nätter/helger, detta är mycket smidigt då vi är rumskamrater.\\
\vspace{11pt}
Då spelet är modulärt är det lätt att dela upp arbetet, där man jobbar på separata moduler. Vi kommer även läsa/kommentera/testa koden som den andre skrivit så vi får full förståelse för hela spelet och fångar upp eventuella buggar och ineffektiv kod.\\
\vspace{11pt}
Betygsambitionerna är ganska höga, men det spelar ingen större roll om vi får en 3:a p.g.a. andra ämnen och intressen.\\
